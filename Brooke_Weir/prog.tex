\documentclass[10pt,draftclsnofoot,onecolumn,letterpaper]{article}
\usepackage{ragged2e}
\usepackage[svgnames,table]{xcolor}
\usepackage[hidelinks]{hyperref}
\usepackage{listings}
\usepackage{graphicx}
\usepackage{longtable}

\usepackage[utf8]{inputenc}
\usepackage[left=0.75in, right=0.75in, top=0.75in]{geometry}
    
\usepackage[T1]{fontenc}
\setlength{\parindent}{0pt}
    
    
\begin{document}
\begin{Center}
{\fontsize{14pt}{16.8pt}\selectfont OPEnS Pyranometer: IoT Solar Radiation Sensor\\ Progress Report Winter Term\par}
\end{Center}\par
    
\begin{Center}
Garen Porter, Brooke Weir, Alejandro Tovar\\
\end{Center}\par
\hrule
\begin{Center}
{\fontsize{12pt}{16.8pt}\selectfont \textbf{Abstract}\par}
\end{Center}\par

TBD
    
{\fontsize{10pt}{12.0pt}\selectfont 
\par}\par
\newpage
\pagenumbering{arabic}
\tableofcontents
\clearpage
    
\section{Goals}
    The goal of the Pyranosaurs team is to make an IoT solar radiation sensor that is open-source, low-cost, accurate, and easy to assemble. Open-source pyranometers exist, but they are poorly documented and not wireless. To keep the pyranometer low-cost, the parts used will either be inexpensive or 3D printed. To make the pyranometer open-source, the CAD designs and code will be freely available online to give users the ability to construct the pyranometer anywhere with access to a Maker Space. Instructions to build, operate, and deploy the pyranometer will be thorough and easy to follow to make the pyranometer as user friendly as possible. As the pyranometer project progresses, it will continually be tested against commercial-grade sensors to verify accuracy. Due to the pyranometer project being sponsored by the OPEnS lab, it must successfully integrate into the LOOM IoT system and follow the same protocols as the other OPEnS lab projects. As with other OPEnS projects, the pyranometer will be wireless. To do this, the pyranometer will transmit data via LoRa radio to a data hub that will use a PushingBox API that translates the data into HTTPS and uploads the data to a Google sheet.
    
\section{Current State}
TBD

\section{Problems and Solutions}
TBD

\section{Interesting Code}
The below code segment is the setup function for the Adafruit TMP007, better known as the thermopile. The thermopile is registered under I2C address 0x41, which the multiplexer uses to interact with the thermopile. The delay value is the interval at which the thermopile takes readings, this value is adjusted per the thickness of the black body object being measured.
\begin{lstlisting}[caption={TMP007 Setup},captionpos=b]
bool setup_tmp007() 
{
    //Setup Here
    Serial.print("SETTING UP THERMOPILE\n");
	bool is_setup;
	state_tmp007.inst_tmp007 = Adafruit_TMP007(0x41);
	if(state_tmp007.inst_tmp007.begin()){
		is_setup = true;
		config_tmp007.delay = 4000;
		LOOM_DEBUG_Println("Initialized tmp007");
	}
	else{
		is_setup = false;
		LOOM_DEBUG_Println("Failed to initialize tmp007");
	}

    Serial.print("FINISHED SETTING UP THERMOPILE\n");
    return is_setup;
}
\end{lstlisting}

Listing 2 shows the thermopile taking measurements. The thermopile reports the local temperature of an object (the black body), the temperature of the thermopile die, and the voltage generated by the thermopile. After this function is called by the multiplexer, the data is packed into an OSC bundle and either written to an SD card or transmitted over a wireless protocol (Wifi or LoRa). 

\begin{lstlisting}[caption={TMP007 Setup},captionpos=b]
void measure_tmp007() 
{
  Serial.print("MEASURING\n");
	state_tmp007.volt = state_tmp007.inst_tmp007.readRawVoltage();
	state_tmp007.obj_temp = state_tmp007.inst_tmp007.readObjTempC();
	state_tmp007.die_temp = state_tmp007.inst_tmp007.readDieTempC();

  #if LOOM_DEBUG == 1
    Serial.print(F("[ ")); Serial.print(millis()); Serial.print(F(" ms ] "));
    Serial.print(F("Volts: ")); Serial.print(state_tmp007.volt); Serial.print(F("  "));
    Serial.print(F("Die Temp: ")); Serial.print(state_tmp007.die_temp); Serial.print(F("  "));
    Serial.print(F("Object Temp: ")); Serial.print(state_tmp007.obj_temp); Serial.print(F("  "));
  #endif

  Serial.print("DELAYING\n");
  delay(config_tmp007.delay);
}
\end{lstlisting}

\section{Week by Week Summary}
\subsection{Week 1}
This week we rejoined as a team and decided on a meeting time for our group with our client, Chet Udell. We checked on parts we ordered from the lab in October. Apparently only have of those parts were actually ordered so we were missing parts. We ordered those parts this week. We also discussed what more needed to be done such as testing and building other prototypes. We began working on 3D printing.

\subsection{Week 2}
This week we met with Chet and had our first TA meeting. We had our first 3D printed prototype of the base of the housing structure this week. We tried to get in contact with John Selker to use some of his equipment for testing the prototypes. We ran into an issue with integrating the thermopile into LOOM and will work with Luke next week to fix the issue. We also began working on the poster for expo. 
 
\subsection{Week 3}
This week we didn't get to meet Chet, our client, at our regularly scheduled meeting time. It turns out he has been pulled into a meeting at that time. We changed our weekly meeting time so he can make them happen more often since that meeting was going to be a weekly thing. This week, we got the thermopile integrated with loom. It is now sensing data and putting it on an SD card so we can begin testing which we started to get equipment to do so. The 3d printed housing structure had also gone through several iterations. We continue to try to get in contact with John Selker to see if we can use some of his equipment to test the lux sensor. We also met with our TA this week. We also made a draft for the expo poster and did some early attempts at an elevator pitch.

\subsection{Week 4}
This week we met with our client as well as our secondary client Chad Higgins. Both were pleased with our progress and gave us tips on how to make our design better. Chad also gave us an industry grade radiometer to help test our sensor, specifically the thermopile. However, he said it might be hard to find documentation to make the radiometer work. We also ordered more parts such as a UV sensor and a smaller acrylic dome to make the inner dome of the double dome. We also ordered parts for LoRA transmission so that when we are ready to start transmitting our data via radio, we will have the parts. We also got in contact with a group that is using a fluke thermometer and we will work with them to test the thermopile as well. 

\subsection{Week 5}
This week, we met with Chet and our TA as usual. We had more parts ordered like a glass square to fit over the UV sensor and we received the smaller acrylic domes we ordered. We had an opportunity to test the thermopile with the fluke thermometer, but we encountered problems getting data off the instrument. We started another print of our housing structure based on the feedback from last week. 

\subsection{Week 6}
This week we continued to work in the lab to get the fluke thermometer to test the thermopile. We weren't able to get data off the fluke thermometer, but visually it looked like the thermopile was within a few degrees of the fluke thermometer. We also soldered the UV sensor and tried to use the example code to get the sensor to read data. That didn't work, but we read that the UV sensor sometimes only works with another sensor. So we integrated it into LOOM in hopes that it will work with another sensor. The 3D print of the housing structure continued this week. This is a good prototype for base functionality of the housing structure. We were able to finally get in contact with John Selker and get a silicon pyranometer to help test our sensors. This week we did not get to see Chet at our monthly meeting, but we did see our TA. 

\subsection{Week 7}
TBD

\subsection{Week 8}
TBD
 
\subsection{Week 9}
TBD

\subsection{Week 10}
TBD

\section{Retrospective}
%\begin{center}
    \begin{longtable}{| p{0.17\linewidth} | p{0.22\linewidth} | p{0.22\linewidth} | p{0.22\linewidth} |}
         \hline \textbf{Task} & \textbf{Positives} & \textbf{Deltas} & \textbf{Actions} \\
         \hline Problem Statement & Completed. & None. & None. \\
         \hline Requirements & Completed. & Continue to revise. & Change based on further comments from Chet. \\
         \hline Design Document & Completed. & Continue to revise. & Make changes based on client feedback. \\
         \hline Feather M0 & Works with Adalogger and sensors; able to write sensor data to an SD card. Connected to a multiplexer (Adafruit TCA9548A).  & Be able to get readings from multiple sensors, be powered wirelessly, and transmit data over LoRa. & Wire a lithium-ion battery pack to the feather. \\
         \hline Lux sensor & Able to sense and put data on an SD card; integrated into LOOM. & Verify accuracy of reported data. & Test the lux sensor probably against standard IR sensor. \\
         \hline Thermopile & Able to sense data and put it on an SD card via LOOM and a mulitplexer. & Verify accuracy of reported data. & Test the theropile against other sensors. \\
         \hline UV Sensor & Built and integrated into LOOM, but data is not being read. & Get the sensor to report data & See if integrating the sensor works. \\
         \hline Photodiode & We understand how the photodiode method works. & Need the parts to arrive; need to construct a photodiode based pyranometer. & Start building a prototype. \\
         \hline LOOM & We understand the purpose and function of LOOM; we understand how new sensors can be integrated into LOOM. We have integrated a thermopile and a UV sensor. & Transmit data online. & Work with OPEnS staff to understand how LOOM transmits data via LoRA. \\
         \hline Housing structure & A good proof of concept prototype has been printed.  & Needs to be printed for the final version on the better printer once it is fixed. & Talk to OPEnS staff to see when the printer is fixed.  \\
         \hline Weatherproofing & Researched ways to insulate against wind and protect the sensor from water damage. & Needs to be applied to structure. & Build the structure and apply ideas that have been researched. Test the housing structure for damage before putting electronics inside. \\
         \hline Black Body Object & The best black body object to use is a common 3D priting material & Create a black dome made out of PLA. & Design the dome using CAD and print it. \\
         \hline Power Source & The lithium-ion battery chosen to be used is the same power source used in many other OPEnS projects. & Make the pyranometer wireless & Wire a lithium-ion battery pack to the Feather M0. \\
         \hline LoRa Radio & Can be implemented easily with LOOM using the LOOM API. & Give the pyranometer the ability to trasmit data over LoRa. & Wire a LoRa chip to the Feather M0, attach an antennae, and usee LoRa LOOM API to trasmit data readings to a Google sheet. \\
         \hline
    \end{longtable}
%\end{center}

\end{document}